% [H] means put the figure HERE, directly when you input this code.
\begin{figure}[H]
	\centering
	
% We use a figure width of 48.5% of the width of one line of text on 
% the page so there is some space between the images.
	\subfloat[Left image sub-caption.]{
		\includegraphics[width=0.485\linewidth]{./graphics/dragon.png}\label{fig:example_2x1_a}
	}~ % Use a tilde to add spacing for sub-figures that are displayed next to one another horizontally.
	\subfloat[Right image sub-caption.]{
		\includegraphics[width=0.485\linewidth]{./graphics/dragon.png}\label{fig:example_2x1_b}
	}\\ % New line before caption.

% Caption is defined with a short and long version. The short version is shown in the 
% List of Figures section, and the long version is used directly with the figure. 	
	\caption[A demonstration of a 2x1 sub-figure layout.]{Construct sub-figures from multiple image files in LaTeX not in the image file itself. This allows you to tweak the positioning and layout without having to modify the images. It also allows for automatic formatting and numbering of captions and sub-captions. Image of glass dragons rendered using Path Tracing \cite{whittle15_dragons}.}
	
% For figures label should be defined after the caption to ensure proper figure numbering.
	\label{fig:example_2x1}
	
\end{figure}