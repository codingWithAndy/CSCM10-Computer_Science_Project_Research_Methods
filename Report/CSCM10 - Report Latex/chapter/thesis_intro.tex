\chapter{Introduction}
	\label{chap:intro}
	
	This document is intended both as a thesis template and a written tutorial on typesetting a professional looking academic document. The style of the template is designed to mimic an equivalent LaTeX document template that is commonly used for within the Computer Vision and Visual Analytics group here at Swansea. This LaTeX template is itself based on a LaTeX template named Custard. 
	
	\section{Motivations}
		\label{sec:intro_motivation} 
		
		Large documents can become cumbersome to work with and format consistently. Sensibly chosen aesthetic cues are important to help imply structure and can greatly aid the reader in understanding your work. The accompanying LaTeX template uses abstraction to hide the formatting from the author during content preparation, allowing for consistent styling to be applied automatically during document compilation. In this Google Docs theme it is the responsibility of the author to manually adhere to the styling laid out in this template.
	
	\subsection{Objective}
		\label{sec:intro_objective} 
		
		In this document we present a tutorial on thesis creation and typesetting, and discuss topics such as literature surveying and proper citation. 
		
	\section{Overview}  
		\label{sec:intro_overview} 
		
		The remainder of chapter \ref{chap:intro} outlines the document structure and the key contributions of this work is organized as follows. Chapter \ref{chap:resources} reviews techniques for finding and properly citing external resources from the academic literature and online. In chapter \ref{chap:typesetting} we show examples of how to typeset different types of content, such as internal references, figures, code listings, and tables. And lastly in chapter \ref{chap:conclusion} we summarize the main contributions and key points to take away from this template.
	
	\section{Contributions} 
		\label{sec:intro_contribs} 
		
		The main contributions of this work can be seen as follows:
		
		\begin{description}	
		
			\item[$\bullet$ A LaTeX thesis template]\hfill
			
			Modify this document by adding additional TeX files for your top level content chapters. 
			
			\item[$\bullet$ A typesetting guide of useful primitive elements]\hfill
			
			Use the building blocks within this template to typeset each part of your document. Aim to use simple and reusable elements to keep your LaTeX code neat and to make your document consistently styled throughout.
			
			\item[$\bullet$ A review of how to find and cite external resources]\hfill
						
			We review techniques and resources for finding and properly citing resources from the prior academic literature and from online resources.
			
		\end{description}