\chapter{Introduction}
	\label{chap:intro}
	
	This report aims to research the concept of gamification. Exploring what it is, by exploring its background and history, how education uses gamification, as well as how it aids fundamental teaching concepts. Also, focusing on how gamification, within a science context, has been applied. With the aim of the findings creating a foundation for the author’s masters dissertation.
	
	
	\section{Project Description}
		\label{sec:intro_pro_desc} 	
		
		
		The masters' thesis will be about creating a game about \ac{ML}. Through using multiple libraries within Python, which helps to develop \ac{ML} programs through such libraries as SciKit Learn, and by using Pygame, which helps develop game-like features while using Python, rather than the more traditional languages, such as C or C++. The aim will be to create a program that will allow players to interact with and, through playing, learn the mechanics of \ac{ML}. 
		
		The main aim of the game will be to teach \ac{ML} concepts, while at the heart of the program, applying \ac{ML} to the program to create the core game. For example, the player will have to plot data points onto a game area of pre-populated data points. The player will then have to plot where they think the decision boundary is for that model. Once the player places where they think the boundary is, the model will score how accurate they were, rewarding a score depending on their accuracy. Therefore, in order for the player to do well within the game, they will have to know how the different \ac{ML} models work. Areas within the game will explain the different models, allowing the player to understand the fundamentals of the model, with additional features outside of the game to help them learn more about \ac{ML}.
		
	\section{Motivations}
		\label{sec:intro_motivation} 
		
		In the age of big data, \ac{ML} and \ac{AI} have become a big part of peoples every day lives. “It is predicted that by 2025, the global \ac{AI} market is expected to be almost \$60 billion; in 2016, it was just \$1.4 billion. Another research says that \ac{AI} bots will drive up to \$33 trillion of annual economic growth and also power 85\% of customer service interactions by 2020 \cite{hackernoon}.” However, many people perceive \ac{ML} to be a black box, a form of sophisticated computer magic. Nevertheless, \ac{ML} is a complex technique to master; the underlining factors of \ac{ML} are relatively simple, to a degree.
		
		With the intentions to try and debunk this myth and misconceptions around \ac{AI} and \ac{ML}, creating a tool that can help educate people into the concepts of \ac{ML} would be a valuable commodity. "Despite popular opinions, games promote learning and discourage negative behaviours. One study illustrates that regular gameplay improved mental health as well as cognitive and social skills \cite{classcraft}.” With this in mind, creating a teaching and learning tool that incorporates gamification features would not only bring a source of enjoyment to people and players but also create interactive ways to keep the user engaged. In turn, aiming to help change the concept of \ac{ML} from being an unknown black box into a well understood and embraced tool.
	
	\subsection{Objective}
		\label{sec:intro_objective} 
		
		In this report, we explore what gamification is, what it consists of, and how it can aid and reinforce the teaching and learning. We will also explore techniques and features used within gamification that will aim to keep the product engaging as well as incentivise the user to return.
		
	%\section{Overview}  
%		\label{sec:intro_overview} 
%		
%		The remainder of chapter 1 outlines the document structure. This report contains chapter 2, which, researches into the background of gamification and its history. This chapter also reflects on have gamification is used within a Science context. The chapter also looks at researching potential ways of applying gamification techniques to science concepts/ideas. An example of this is through using potential appropriate user engagement and incentivisation techniques teaching key machine learning concepts. Also on the impact gamification has had within education, the classroom and beyond. We are looking at how it can aid teaching while reinforcing learning and attainment levels. Furthermore, lastly in chapter 3, we summarise the main contributions and main concepts into gamification.
	
%	\section{Contributions} 
%		\label{sec:intro_contribs} 
		
%		The main contributions of this work can be seen as follows:
		
%		\begin{description}	
		
%			\item[$\bullet$ Background into Gamification]\hfill
			
			 
			
%			\item[$\bullet$ Gamification of Science]\hfill
			
%			Use the building blocks within this template to typeset each part of your document. Aim to use simple and reusable elements to keep your LaTeX code neat and to make your document consistently styled throughout.
			
%			\item[$\bullet$ Applying Gamification Techniques to Science Concepts ]\hfill
						
%			We review techniques and resources for finding and properly citing resources from the prior academic literature and from online resources.
			
%			\item[$\bullet$ Gamification Integration with Teaching and Learning ]\hfill
			
%			We review techniques and resources for finding and properly citing resources from the prior academic literature and from online resources.
			
%		\end{description}