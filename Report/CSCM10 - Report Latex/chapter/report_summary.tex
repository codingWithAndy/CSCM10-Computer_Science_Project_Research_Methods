\chapter{Conclusion}
\label{chap:conclusion}

%In this document we have demonstrated the use of a LaTeX thesis template which can produce a professional looking academic document. 

%\section{Contributions} 
%\label{sec:conclusion_contributions}

	Gamification is a powerful tool for engagement. Since its initial conception in 2002 and then Jane McGonigal's groundbreaking TED Talk in 2010, gamification has become a vast field, with it going viral in 2016 with the release of Pokémon Go. CHI conference offered a workshop around gamification, which spawned the GRN in 2011. 
	
	Gamification takes advantage of extrinsic and intrinsic motivations, aiming to enhance our daily activities or tasks. Therefore, in order for the gamification to be most effective, then both these motivation factors need to be accounted for within the task. In order for the person to feel good about oneself, a form of reward has to exist. This feel-good factor is due to the release of dopamine within the body. Which, therefore creates an incentive for the person to do that action again. The mechanics of gamification is tapping into the release of the dopamine. Through using game components like points, badges, leaderboards, performance graphs, avatars and teammates within non-traditional gaming settings, this allows the player to get a sense of competition with themselves, to do better, or with other people.  
	
	All types of gamification work well with science concepts, but the main one used is serious games. These types of games have a primary goal in mind when being designed and are suited for outreach, teaching and research goals.
	
	Gamification in education is fundamentally the same thing as normal gamification, but with more of an emphasis on the learning. However, whether gamification in education has to be done within a non-digital world or not, splits the academic community. Academics agree though that gamification within education has a positive effect on learning, allowing students to have a sense of ownership of their learning.
	
	Including gamification elements within the final masters' thesis, should help create a positive experience for the user. Using mechanics to encourage them to continue playing as well as being able to teach the user key Machine learning concepts, to help educate them and debunk the misunderstanding of what is machine learning, all within a fun and interactive encounter. 
%The main contributions of this work can be summarized as follows:
%\begin{description}	

%	\item[$\bullet$ A LaTeX thesis template]\hfill
%	
%	Modify this document by adding additional top level content chapters. These descriptions should take a more retrospective tone as you include summary of performance or viability. 
%	
%	
%	\item[$\bullet$ A typesetting guide of useful primitive elements]\hfill
%	
%	Use the building blocks within this template to typeset each part of your document. Aim to use simple and reusable elements to keep your document neat and consistently styled throughout.
%	
%	\item[$\bullet$ A review of how to find and cite external resources]\hfill
%		
%	We review techniques and resources for finding and properly citing resources from the prior academic literature and from online resources. 
	
%\end{description}

%\section{Future Work}
%\label{sec:conclusion_future_work}

%Future editions of this template may include additional references to Futurama.