\chapter{Background into Gamification}
	\label{chap:resources}
		
	% -- does gamification history need to go here? ---------  
	%The university has subscriptions to a vast number of major academic journals spanning a wide range of subject areas. By accessing the internet from a university network connection (Eduroam or Ethernet), the paywalls of many journals will simply vanish without any need for login credentials.

	\section{History of Gamification}
		
		Gamification is known as a powerful tool for engagement, which has since it is initial conception has now become a standard feature within software development [3]. The term gamification first appeared in the context of software design in 2008 [4], but the term only started to get more widespread recognition within 2010. However, the term ‘gamification’ was first coined by Nick Pelling in 2002 [3]. Its initial aim was to incorporate social and reward features of games into the software. Gamification started to gain much attention, so much so that it described by a venture capitalist as one of the most promising areas of gaming [5].
		
		Researchers consider gamification to be the progression of earlier work that focuses on adopting game-design elements to non-game situations and contexts. Research in the human-computer interaction field that uses game-driven elements for motivation and interface design suggests that there is a connection between Soviet concepts of socialist competition and the American management trend of ‘fun at work’ [5].
		
		In 2010, Jane McGonigal delivered a groundbreaking TED Talk titled, ‘Gaming Can Make a Better World’ [9]. This talk is considered the defining moment in the history of gamification. Within the talk, she prophesies a game based paradise. Where she states that “When I look forward to the next decade, I know two things for sure: that we can make any future we can imagine, and we can play any games we want, so I say: Let the world-changing games begin.[9]” Which it could inform she was right, as, from 2011, gamification starts to pick up steam. During this year, at a CHI (Computer-Human Interaction) conference, a workshop titled “Gamification: Using Game Design Elements in Non-Gaming Contexts [10]”, which spawned the Gamification Research Network (GRN) [11]. Through the years 2012 to 2016, gamification continues to grow. Even so, that gamification goes viral without people knowing through a game called Pokémon Go. Pokémon Go is one of the most successful applications of gamification with over 800 million downloads. People who would usually turn their nose up at badge collection were out patrolling the streets searching for rare Pokémon! Pokémon Go is one of the most successful apps of all time. It even broke records [12, 3]. It could be said thanks to Pokémon Go, that gamification is everywhere.
		
		Many established technology and other companies, including SAP AG, Microsoft, IBM, SAP, LiveOps, Deloitte, and other companies have started using gamification in various applications and processes [6].
		
		The increased popularity in gamification, within some contexts, has had led to many legal restrictions be placed upon it. However, this mainly refers to the use of virtual currencies and assets, as well as data privacy, data protection and labour laws. These laws are due to its nature of being a data mining systems that spread information online, known as data aggregator [7, 8].

	\section{The Science of Gamification}
		\label{sec:google_fu}
		
		Games are fun, and there is no denying that whether it is playing more traditional video games, mobile games or a recent phenomenon Mcdonalds Monopoly. The games industry is work an estimated \$2.3 trillion, show that the global entertainment and media business is massive everywhere [13]. There is a reason behind this, as games made are crafted with the human brain in mind. From each roll of the dice, getting the correct combination, to defeating an opponent and enemy, to building a new settlement, each action rewards to bain and its reward centre lights up [14].
		
		By incorporating aspects from games like points, levels and progression bars into non-game situations, we can recreate the experience of gaming. Having these elements within a product, to interact with the user, is why gamification is so powerful.
		
		Games ranging from Super Mario Bros. to Monopoly have a real impact on brains and the way we learn. These impacts on are brain are due to dopamine. Dopamine is a neurotransmitter within a person’s brain that is triggered within a person whenever we do something positive or when a person feels that they have achieved something [15]. In essence, dopamine is a natural drug that makes people feel good [14]. This drug, dopamine, is an integral part of our learning through reinforcement learning. As Nestler Lab explains, “activation of the pathway tells the individual to repeat what it just did to get that reward [14, 16].” We do something well, and we get a sense of reward from our brains which leads us to do it again. Hence why we as humans tend to feel good when we are learning something; however, it is not very easy to stay motivated while learning as the skill factor/learning requirements increases. At this stage is where gamification shines and can help keep the user/learner motivated with a little boost along the way. The motivation, the critical factor gamification tries to manipulate, is triggered by the sense of success. Which leads onto more willingness and desire to do something, can be achieved by not only rewarding the final goal but by also releasing small amounts of dopamine as we are edging closer to a goal. Allowing a user to know if they are nearing a milestone can be achieved by using progress bars, as they get closer to the end goal. Each sub-goal completed filling up the bar giving instant gratification, with small hits of satisfaction and dopamine, on the build-up to meeting the primary goal and that massive hit of dopamine, therefore creating that motivation to keep going. This situation becomes superseded only when an unexpected gratification situation occurs, releasing even more dopamine.
		
		While motivation is at the centre of gamification, our enthusiasm comes from three main areas: Autonomy; Value; Competence [17]. If someone is in charge of their destiny, they are more motivated to succeed. Allowing the person more control will mean that they will work harder towards objectives, especially for a more extended period, when given the opportunity and authority to select their direction when solving a problem. This aspect is giving them autonomy. The second principal area value is about the person feeling value to an activity or action. If the person feels that there is self-worth to the activity, then they will increase interest in the activity and increase their motivation levels. Research states that a positive correlation occurs when a student values a subject at school and their willingness to investigate a question. If the person cares, they will keep going and work harder until the task is completed [13?/17?]. Finally, the third area is competence. If a person develops a certain degree of proficiency at something, they are more likely to keep doing it. Another study has shown that there is a link between a student’s sense of mastery and their desire to continue certain activities. Those who credit natural talent rather than hard work will more likely give up more quickly.
		
		Gamification aims to take advantage of our extrinsic motivations, factors like final grades or money, and intrinsic motivation, trains like personal gains or enjoyment, to try and enhance our daily activities or tasks. Therefore, in order for the gamification to be most effective, then both these motivation factors need to be accounted for within the task. In order for the person to feel good about oneself, a form of reward has to exist [13?].
		
		%The internet is big \cite{sizeofinternet}. Knowing how to phrase a question to a search engine is therefore an invaluable skill. If the request is simple enough, even a poorly structured query will likely return usable results. For more difficult to find resources you can leverage the language of the search engine to gather relevant papers and resources for your research more efficiently. 
		
		% An example of how to center a passage of text, control local fontsize, 
		% and create a properly formatted and clickable URL.
		%\begin{center}
		%{\small \url{https://www.gwern.net/Search}}
		%\end{center}
		
		%``Internet Search Tips'' \cite{gwern} provides an excellent review of methods and tips for scouring the internet for hard to find resources. You will also be less likely to get caught behind journal paywalls when working remotely without a tunnel as your queries can be made to look for raw pdfs that are often released by the authors directly.
			
	\section{Gamification in Science}
		\label{sec:resources_bibtex}
	
		BibTeX is a language for specifying resource citations. Every time you access and read an academic paper, take code from an online repository, or source the media such as images from existing works you should create a BibTeX entry in a file that you keep throughout your research. Software such as Mendeley \cite{mendeley} can help automate the process of building your BibTeX library of citations. 
		
		\lstinputlisting[label={lst:bibtex}, caption={An example BibTeX entry for an academic paper published in conference proceedings \cite{kaj86}.}]{./listings/example_bibtex.bib}
		
		The BibTeX code listing above (listing \ref{lst:bibtex}) shows an example of how to cite an academic paper, in this case one of the central papers in Computer Graphics research. The key \textbf{kaj86} is an arbitrary name chosen as a meaningful identifier for the resource. In the document text we can call on this resource as an inline citation using the LaTeX command \lstinline|\cite{kaj86}| which produces \cite{kaj86} at the location it is called. As long as a citation has been used at least once somewhere within the document then a formatted full citation will be created in the bibliography at the end of the document with the same citation number that is shown inline.
		
		It is considerably easier to be disciplined in methodically taking note of the resources you access and make use of as you access them, than it is to try and hunt them all down again at the time you need to write about them in your document. Invest time in being organized and consistent up front and it will be easier when you come to write up.
		
	\section{Gamification in Education}
		\label{sec:gamification_edu}
		Usually you would not put the URL of the resource you are citing directly in the text like is done previously in section \ref{sec:google_fu}. The citation for the resource \cite{gwern} is sufficient to reference it within the text given that full details of its location are then kept neatly within the bibliography at the end of the document. 
		
		In normal usage the purpose of a citation is not to direct the reader away from your thesis, but to justify and back up assertions you are making about the state of the domain. If a reader questions your assertions then they can follow the rabbit hole of papers which will likely also make and justify assertions with even earlier papers from the literature. 
		
		In the above case the intention is for the reader of this template to actually go to that resource and read what it has to say directly. The link is therefore shown clearly within the main text to indicate that the reader should visit it. This as opposed to wanting the reader to purely acknowledge that the facts which are within the resource legitimize the points made in this document, in which case a simple inline citation is the best way to back up your assertions. Section \ref{sec:typesetting_figures_citation} specifically touches on the best practice for how to cite images which you are importing from existing work. 